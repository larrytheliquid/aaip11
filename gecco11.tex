\documentclass{acm_proc_article-sp}

\begin{document}

\title{Verified Stack-Based Genetic Programming via Dependent Types}

\numberofauthors{1}
\author{
\alignauthor Larry Diehl\\
       \email{larrytheliquid@gmail.com}
}

\maketitle
\begin{abstract}
Abstract here.
\end{abstract}

\category{1.2.2}{Artificial Intelligence}{Automatic Programming}[program synthesis, program verification]
\category{D.1.1}{Programming Techniques}{Functional Programming}[dependent types]

\terms{Verification}

\keywords{stack-based gp, linear gp, dependent types, purely functional gp, formal methods}

\section{Introduction}

Genetic Programming (GP) as a field began by using the meta-language
(Lisp) to directly represent algorithmic terms. This is primarily due
to the easy manipulation of parse trees and built-in evaluation
function over them. Naturally, a lot of GP's early and prominent work
was dominated by dynamic language implementations, clever avoidance of
\cite{TO:DO} type-issues during evolution, and tree-based program terms
for the algorithm to operate on.

Since then, there has been work on using type-awareness to inform the
different stages of the algorithm \cite{TO:DO} (population generation,
genetic operators, etc.) There has also been plenty of separate work
on alternative representations (linear, graph, grammar-based, etc.)

\section{Conclusion}


\bibliographystyle{abbrv}
\bibliography{gecco11}

\end{document}
